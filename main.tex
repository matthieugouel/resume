%%%%%%%%%%%%%%%%%%%%%%%%%%%%%%%%%%%%%%%%%
% "ModernCV" CV and Cover Letter
% LaTeX Template
% Version 1.11 (19/6/14)
%
% This template has been downloaded from:
% http://www.LaTeXTemplates.com
%
% Original author:
% Xavier Danaux (xdanaux@gmail.com)
%
% License:
% CC BY-NC-SA 3.0 (http://creativecommons.org/licenses/by-nc-sa/3.0/)
%
% Important note:
% This template requires the moderncv.cls and .sty files to be in the same
% directory as this .tex file. These files provide the resume style and themes
% used for structuring the document.
%
%%%%%%%%%%%%%%%%%%%%%%%%%%%%%%%%%%%%%%%%%

%-------------------------------------------------------------------------------------
%	PACKAGES AND OTHER DOCUMENT CONFIGURATIONS
%-------------------------------------------------------------------------------------

\documentclass[11pt,a4paper,sans]{moderncv} % Font sizes: 10, 11, or 12; paper sizes: a4paper, letterpaper, a5paper, legalpaper, executivepaper or landscape; font families: sans or roman

\moderncvstyle{banking} % CV theme - options include: 'casual' (default), 'classic', 'oldstyle' and 'banking'
\moderncvcolor{blue} % CV color - options include: 'blue' (default), 'orange', 'green', 'red', 'purple', 'grey' and 'black'


\usepackage[scale=0.90,bmargin=1.5cm,footnotesep=0.5cm]{geometry}

\usepackage[american]{babel}
\usepackage[utf8]{inputenc}

\AfterPreamble{\hypersetup{
    colorlinks = true,
    urlcolor = [RGB]{36,34,79},
    linkcolor = gray,
}}

\definecolor{ultramarine}{RGB}{0,32,96}


%-------------------------------------------------------------------------------------
%	NAME AND CONTACT INFORMATION SECTION
%-------------------------------------------------------------------------------------

\firstname{Matthieu}
\familyname{Gouel}

\title{Researcher}
% \address{}{}
% \mobile{}
\email{matthieu.gouel@protonmail.com}
\social[github]{matthieugouel}


%-------------------------------------------------------------------------------------

\begin{document}
% \vspace*{-15mm}
\makecvtitle

%-------------------------------------------------------------------------------------
%	WORK EXPERIENCE SECTION
%-------------------------------------------------------------------------------------
% \vspace*{-10mm}
\section{Experience}

\vspace*{+2mm}
\cventry{2023 - Present}{Research Fellow}{Cloudflare}{Paris, France}{}{
\vspace*{+1mm}
\textcolor{ultramarine}{
    Embarking on a three-month program with Cloudflare's research team.
    Analyzing network measurement data to enhance the performance and reliability of Cloudflare's products.
}
\vspace*{+1mm}}

\cventry{2020 - 2023}{Ph.D. Student}{Sorbonne Université}{Paris, France}{}{
\vspace*{+1mm}
\textcolor{ultramarine}{
    Ph.D. student in the Network Performance Analysis (NPA) team at the LIP6 lab, also affiliated with LINCS lab, under the guidance of Dr. Olivier Fourmaux and Dr. Timur Friedman. My research focused on analyzing internet features through active and passive measurements. This included studying the IP-level, router-level, and AS-level topology, as well as geolocating infrastructure IP addresses.
}
\vspace*{+1mm}
\begin{itemize}
\item Development of a multi vantage-points measurement platform for IP-level internet topology discovery using \textbf{Python}, \textbf{ClickHouse}, \textbf{Redis}, \textbf{JavaScript} and \textbf{AWS/GCP} cloud infrastructures.
\item Development of a \textbf{reinforcement learning algorithm} to allocate probing directives more efficiently.
\item Development of a framework for large-scale traceroute processing.
\item Analysis of the dynamics of an IP geolocation database.
\item Study of \textbf{topology dynamics} and \textbf{measurement ethics} at Naval Postgraduate School (Monterey, CA).
\item Co-organizer of LINCS Internet Measurements reading group.
\end{itemize}}

%------------------------------------------------

\vspace*{+2mm}
\cventry{2019 - 2020}{Research Engineer}{LIP6}{Paris, France}{}{
\vspace*{+1mm}
\textcolor{ultramarine}{
    Worked in the Network Performance Analysis (NPA) team of the LIP6 laboratory and specifically in the Dioptra research group.
    Helped in the engineering of ongoing research projects in \textbf{Python} and \textbf{C++},
    developed and maintained measurement infrastructure in a production environment
    using \textbf{Docker}, \textbf{Kubernetes} and \textbf{GitHub Actions}.
}
\vspace*{+1mm}
\begin{itemize}
\item Participated in ongoing researches about IP-level internet topology, IP geolocation and Alias Resolution.
\item Building and maintaining the network measurement infrastructure.
\end{itemize}}

%------------------------------------------------

\vspace*{+2mm}
\cventry{2018 - 2019}{DevOps Engineer}{Société Générale}{Paris, France}{}{
\vspace*{+1mm}
\textcolor{ultramarine}{
    Worked inside an agile software development team for a global private cloud development project.
    This team was responsible for network-oriented capabilities of the cloud.
    I was the lead developer of the team and built APIs in \textbf{Python} which communicate with Avi Networks appliances.
    I also developed and open-sourced a \textbf{Python} package to facilitate API end-to-end testing.
}
\vspace*{+1mm}
\begin{itemize}
\item Development of APIs in \textbf{Python} providing SDN capabilities for the internal private cloud.
\item CI/CD of programming projects using \textbf{Github Actions}, \textbf{Ansible} and \textbf{Jenkins}.
\item Organized according to Scrum and SAFe agile methodologies.
\end{itemize}}

%------------------------------------------------

\vspace*{+2mm}
\cventry{2016 - 2018}{Network-oriented Software Engineer}{French Ministry for Armed Forces}{Paris, France}{}{
\vspace*{+1mm}
\textcolor{ultramarine}{
    Worked in a team managing network infrastructure for Data-Centers and Campus.
    My main role was to bring development skills to automate network infrastructure deployment and  maintenance in operational conditions.
    I also participated to networking troubleshooting and on-call duty periods.
}
\vspace*{+1mm}
\begin{itemize}
\item Development of network automation software in Data-Center and Campus networks in \textbf{Python}, \textbf{Go} and \textbf{JavaScript}.
\item Deployment with \textbf{Docker}, \textbf{Kubernetes} and \textbf{Gitlab CI}.
\item Network administration and development in production environment.
\end{itemize}}

%------------------------------------------------
\newpage
\section{}  % Separator between jobs and internships
%------------------------------------------------

\vspace*{+2mm}
\cventry{2016}{Network-oriented Software Intern}{French Ministry for Armed Forces}{Paris, France}{6 month}{
\vspace*{+1mm}
\textcolor{ultramarine}{
    Internship at the end of the engineering school.
    The project was to develop a proof-of-concept networking solution to do routing in Data-Centers with BGP on Cisco/Arista hardware.
    The solution had to be automated using \textbf{Python}, \textbf{Ansible} and Zero-Touch provisioning.
}
\vspace*{+1mm}
\begin{itemize}
\item Automation of a BGP "Top of Rack" solution into Data-Centers.
\end{itemize}}

%------------------------------------------------


\cventry{2015}{Software Intern}{EBRC}{Luxembourg}{3 month}{
\vspace*{+1mm}
\textcolor{ultramarine}{
    Internship during the second year of engineering school.
    The project was to develop a knowledge system to help security operators to keep track of internal procedures.
    Developped in \textbf{Python} and \textbf{JavaScript}.
}
\vspace*{+1mm}
\begin{itemize}
\item Development of a  knowledge database system for security operations in a Security Operation Center.
\item Integration in a production environment.
\end{itemize}}

%-------------------------------------------------------------------------------------
%	EDUCATION SECTION
%-------------------------------------------------------------------------------------

% \newpage
\section{Education}

\cventry{2020--2023}{\href{https://www.sorbonne-universite.fr/en}{Sorbonne Université}, Paris, France}{Ph.D.}{Computer Science}{}{}
\cventry{2013--2016}{\href{https://telecomnancy.univ-lorraine.fr/?lang=en}{Télécom Nancy}, Nancy, France}{\href{https://en.wikipedia.org/wiki/Grande_école}{Masters degree} (Diplôme d'ingénieur)}{Networks and Security}{}{}
\cventry{2010--2013}{Blaise Pascal School, Rouen, France}{\href{https://en.wikipedia.org/wiki/Classe_préparatoire_aux_grandes_écoles}{Preparatory Classes}}{
Physics, Technology and Engineering Science}{}{}

%-------------------------------------------------------------------------------------
%	PUBLICATION SECTION
%-------------------------------------------------------------------------------------

\section{Publications}

\href{https://hal.science/hal-04215113/document}{\textbf{Replication: Towards a Publicly Available Internet scale IP Geolocation Dataset}}\\
O Darwich, H Rimlinger, M Dreyfus, M Gouel, K Vermeulen\\
ACM Internet Measurement Conference (IMC), 2023

\vspace{0.25cm}
\href{https://hal.science/hal-04218315v1/document}{\textbf{Poster: Towards a Publicly Available Framework to Process Traceroutes with MetaTrace}}\\
M Gouel, O Darwich, M Mouchet, K Vermeulen\\
ACM Internet Measurement Conference (IMC), 2023

\vspace{0.25cm}
\href{https://hal.science/hal-04090683v1/document}{\textbf{Vers des mesures haute fréquence d'internet plus éthiques}}\\
M Gouel, K Vermeulen, H Rimlinger, O Fourmaux, T Friedman\\
CoRes, 2023

\vspace{0.25cm}
\href{https://hal.science/hal-03656974/document}{\textbf{Zeph \& Iris cartographient l’internet}}\\
M Gouel, K Vermeulen, M Mouchet, J P Rohrer, O Fourmaux, T Friedman\\
CoRes, 2022

\vspace{0.25cm}
\href{https://hal.science/hal-03597580/document}{\textbf{Zeph \& Iris map the internet: A resilient reinforcement learning approach to distributed IP route tracing}}\\
M Gouel, K Vermeulen, M Mouchet, J P Rohrer, O Fourmaux, T Friedman\\
ACM SIGCOMM Computer Communication Review (CCR), 2022

\vspace{0.25cm}
\href{https://dl.ifip.org/db/conf/tma/tma2021/tma2021-paper2.pdf}{\textbf{IP Geolocation Database Stability and Implications for Network Research}}\\
M Gouel, K Vermeulen, O Fourmaux, T Friedman, R Beverly\\
Network Traffic Measurement and Analysis Conference (TMA), 2021

\vspace{0.25cm}
\href{https://arxiv.org/pdf/2002.00252.pdf}{\textbf{Alias Resolution Based on ICMP Rate Limiting}} \\
K Vermeulen, B Ljuma, V Addanki, M Gouel, O Fourmaux, T Friedman, R Rejaie\\
Passive and Active Measurement Conference (PAM), 2020

%-------------------------------------------------------------------------------------
%	TEACHING SECTION
%-------------------------------------------------------------------------------------

\newpage
\section{Teaching}

\cventry{2019, 2020, 2021, 2022}{Master's degree program}{Networks Architecture}{Sorbonne Université}{Teaching assistant (40h/year)}{}
\cventry{2019, 2020, 2021}{Master's degree program}{Internet Measurements}{Sorbonne Université}{Teaching assistant (28h/year)}{}
\cventry{2019}{Master's degree program}{Routing in networks}{Sorbonne Université}{Teaching assistant (36h/year)}{}
\cventry{2022}{Licentiate degree program}{Programming elements}{Sorbonne Université}{Teaching assistant (20h/year)}{}

%-------------------------------------------------------------------------------------
%	OPEN-SOURCE SECTION
%-------------------------------------------------------------------------------------

\section{Open Source}

\faGithub~\href{https://github.com/matthieugouel}{https://github.com/matthieugouel}

\vspace{+0.25cm}
\textbf{High-speed internet-scale measurement platform}\\
\href{https://github.com/dioptra-io/iris}{Iris} (in \emph{Python}): allow to perform internet-scale multi vantage-points IP-level measurements (accessible \href{https://iris.dioptra.io/#/}{here}).

\vspace{+0.25cm}
\textbf{Reinforcement learning algorithm}\\
\href{https://github.com/dioptra-io/zeph}{Zeph} (in \emph{Python}): maximize the coverage IP-level topology measurements.

\vspace{+0.25cm}
\textbf{API end-to-end testing}\\
\href{https://github.com/societe-generale/spintest}{Spintest} (in \emph{Python}): package to easily perform functional scenarios to test APIs.

\vspace{+0.25cm}
\textbf{Hobby projects}\\
\href{https://github.com/matthieugouel/gibica}{Gibica} (in \emph{Python}) and \href{https://github.com/matthieugouel/bjorn}{Bjorn} (in \emph{Rust}): toy programming languages interpreters. \\
\href{https://github.com/matthieugouel/alchina}{Alchina} (in \emph{Python}): implementation of classic machine learning algorithms.

%-------------------------------------------------------------------------------------
%	CERTIFICATIONS SECTION
%-------------------------------------------------------------------------------------

% \section{Certifications}

% % \cventry{2017}{XUDZMGNZVXSK}{Machine Learning}{Coursera}{Stanford University}{}
% %------------------------------------------------
% \cventry{2017 - 2020}{CSCO13183181}{Cisco Certified Network Associate Routing and Switching (CCNA)}{Cisco}{}{}
% %------------------------------------------------
% \cventry{2017 - 2020}{99V21CSNE37310}{Certified Stormshield Network Expert (CSNE)}{Stormshield}{}{}
% %------------------------------------------------
% \cventry{2017 - 2020}{99V21CSNA96935}{Certified Stormshield Network Administrator (CSNA)}{Stormshield}{}{}
% %------------------------------------------------

%-------------------------------------------------------------------------------------
%	LANGUAGES SECTION
%-------------------------------------------------------------------------------------

\section{Languages}

\cvitemwithcomment{English}{Fluent}{Full professional proficiency}
\cvitemwithcomment{French}{Native speaker}{Full professional proficiency}
\end{document}
